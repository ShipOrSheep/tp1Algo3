\documentclass[10pt,a4paper]{article}
\usepackage[utf8]{inputenc} % para poder usar tildes en archivos UTF-8
\usepackage[spanish]{babel} % para que comandos como \today den el resultado en castellano
\usepackage{a4wide} % márgenes un poco más anchos que lo usual
\usepackage{caratula}
% \usepackage[left=3cm,right=3cm,bottom=3cm,top=3cm]{geometry}

% Comandos para simbolos matematicos.
\usepackage{amsmath, amssymb, tabularx}

% Comandos para referencias
\usepackage{natbib}

% Comandos para Figuras, Graficos, Tikz etc.
\usepackage{tikz}
\usepackage{epsfig}
\usepackage{pgfplots}
\usepackage{graphicx}
\usepackage{epsfig}
\usepackage{caption}
\usepackage{subcaption}
\usepackage{svg}

% Comandos para teoremas etc.
\usepackage{amsthm}
\newtheorem{theorem}{Teorema}
\newtheorem{lemma}[theorem]{Lema}
\newtheorem{proposition}[theorem]{Proposición}
\newtheorem{remark}{Observación}
\newtheorem{corollary}{Corolario}
% \newproof{proof}{Demostración}

% Comandos para algoritmos.
\usepackage[noend]{algpseudocode}
\usepackage{algorithm}
\algnewcommand{\IfThenElse}[3]{% \IfThenElse{<if>}{<then>}{<else>}
\State \algorithmicif\ #1\ \algorithmicthen\ #2\ \algorithmicelse\ #3}
\algnewcommand{\IfThen}[2]{% \IfThenElse{<if>}{<then>}
  \State \algorithmicif\ #1\ \algorithmicthen\ #2}

\begin{document}

\titulo{TP1: Optimizando Jambo-tubos}

\subtitulo{}

\fecha{\today}

\materia{Algoritmos y Estructuras de datos III}

\integrante{Oshiro, Javier}{715/09}{javieroshiro@hotmail.com}

\maketitle

\tableofcontents

\newpage

\setcounter{page}{1}

\section{Introducción} \label{sec:introduccion}

\section{Fuerza Bruta} \label{sec:fuerza_bruta}

\section{Backtracking} \label{sec:backtracking}

\paragraph{Poda por factibilidad}

\paragraph{Poda por optimalidad}

\section{Programación Dinámica} \label{sec:dp}

\paragraph{Memoización}

\section{Experimentación} \label{sec:experimentacion}

\subsection{Métodos}

\subsection{Instancias}

%EXPERIMENTOS
%\subsection{Experimento 1}
%\subsection{Experimento 2}
%\subsection{Experimento 3}

\section{Conclusiones} \label{sec:conclusiones}

\end{document}
